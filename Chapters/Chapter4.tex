% Chapter Template

\chapter{Ensayos y resultados} % Main chapter title

\label{Chapter4} % Change X to a consecutive number; for referencing this chapter elsewhere, use \ref{ChapterX}

%----------------------------------------------------------------------------------------

%\section{Introducción}

En este capítulo se presentan las pruebas y ensayos realizados para validar el correcto funcionamiento del dispositivo. También, se incluyen los resultados obtenidos y una tabla comparativa de características técnicas con los dispositivos disponibles en el mercado internacional citados en el capítulo \ref{Chapter1}.

%----------------------------------------------------------------------------------------

\section{Pruebas unitarias}
\label{sec:pruebasU}

Se hicieron pruebas unitarias sobre las bibliotecas desarrolladas para el manejo de los circuitos integrados DS3231, AT24C32 y SX1278. Se utilizó Ceedling para ejecutar dichas pruebas en combinación con Gcov para generar los análisis de cobertura correspondientes. En la tabla \ref{tab:resultsCeedling} se pueden observar los resultados de las pruebas unitarias y en la tabla \ref{tab:coverAnalysis} se exhiben los análisis de cobertura.

\begin{table}[h]
	\centering
	\caption[Pruebas unitarias]{Tabla de resultados de las pruebas unitarias}
	\begin{tabular}{l c c c}    
		\toprule
		\textbf{Biblioteca} & \textbf{Cantidad de tests} & \textbf{Exitosos} & \textbf{Fallidos}  \\
		\midrule
		AT24C32 & 8	& 8 & 0 \\		
		DS3231 & 11 & 11 & 0 \\
		SX1278 & 14 & 14 & 0 \\
		\bottomrule
		\hline
	\end{tabular}
	\label{tab:resultsCeedling}
\end{table}

\begin{table}[h]
	\centering
	\caption[Análisis de cobertura]{Tabla de resultados del análisis de cobertura}
	\begin{tabular}{l c c}    
		\toprule
		\textbf{Archivo} & \textbf{Líneas ejecutadas} & \textbf{Funciones ejecutadas}  \\
		\midrule
		at24c32.c & 52/52 & 6/6 \\		
		ds3231.c & 54/62 & 11/13 \\
		sx1278.c & 172/220 & 26/31 \\
		\bottomrule
		\hline
	\end{tabular}
	\label{tab:coverAnalysis}
\end{table}

%----------------------------------------------------------------------------------------

\section{Pruebas funcionales de firmware}
\label{sec:pruebasFW}

Se probaron los módulos DATA LOGGER, LORA COMMUNICATION y WEB SERVER de la capa superior del firmware, APP. Durante el desarrollo del dispositivo, estos módulos de firmware fueron sometidos a una serie de pruebas para garantizar su correcto funcionamiento, de acuerdo con la planificación del trabajo descrita en el capítulo \ref{Chapter2}. Para ejecutar estas pruebas se utilizó el prototipo de pruebas, un cable micro USB y una PC, su interconexión se muestra en la figura \ref{fig:tbFW}.

\begin{figure}[ht]
	\centering
	\includegraphics[scale=1.2]{./Figures/cuadradoAzul.png}
	\caption{Banco de ensayos para las pruebas de funcionamiento de firmware.}
	\label{fig:tbFW}
\end{figure}

Las pruebas consistieron en monitorear que las líneas de código más importantes de cada uno de estos módulos se ejecute correctamente. Par esto, se añadieron líneas de código a los archivos de código fuente de los módulos para imprimir mensajes de verificación o error. El monitoreo se realizó con la ayuda de la herramienta monitor.py incluida en ESP8266\_RTOS\_SDK.


\section{Pruebas de laboratorio}

Estas pruebas tuvieron como objetivo principal utilizar instrumentación especializada para verificar el correcto funcionamiento de los módulos de hardware que componen el prototipo comercial. Asimismo, registrar las especificaciones técnicas que posteriormente servirán para comparar el dispositivo con otros disponibles comercialmente.


%----------------------------------------------------------------------------------------

\section{Pruebas funcionales de la interfaz web}
\label{sec:pruebasWI}

%----------------------------------------------------------------------------------------

\section{Resultado final}
\label{sec:resultadoF}

%----------------------------------------------------------------------------------------

\section{Comparación técnica con dispositivos disponibles comercialmente}
\label{sec:comparacionDDC}

%----------------------------------------------------------------------------------------